%% WARNING: THIS IS A DERIVED (FLATTENED) FILE; DO NOT MAKE
%% SUBSTANTIVE EDITS! EDIT THE ORIGINAL SOURCE FILES INSTEAD.

% flatex input: [falcon-det.tex]
\documentclass[11pt]{article}
\usepackage{fullpage}
\usepackage{microtype}
\usepackage{newtxtext}
\usepackage{enumitem}
\usepackage{xspace}
\usepackage{import}

\usepackage{amssymb,amsmath,amsfonts}
\usepackage[amsmath,amsthm,thmmarks,hyperref]{ntheorem}
\usepackage{mathtools}
\usepackage{hyperref}
\usepackage[capitalize,nameinlink,noabbrev]{cleveref} % load after hyperref

\usepackage[normalem]{ulem}
\usepackage{url}
\usepackage{breakcites}
\usepackage{xspace}
\usepackage{authblk}

% remove 'draft' to turn off fixme notes
\usepackage[draft,multiuser,inline,nomargin]{fixme}

% fixme: register commands for author(s)
\FXRegisterAuthor{c}{ec}{\color{red}Chris}
\FXRegisterAuthor{d}{ed}{\color{blue}David}
\fxusetheme{color}

% \input{macros}

\begin{document}

\title{Deterministic Falcon Signatures}

% date of last substantive change
\date{November 23, 2021}

\author{David Lazar}
\author{Chris Peikert}
\affil{Algorand, Inc.}

\maketitle

% list of notes to address; appears only in draft mode
\listoffixmes

% flatex input: [overview.tex]

\section{Motivation and Overview}
\label{sec:motivation-overview}

Falcon~\cite{falcon} is a signature scheme that instantiates the
Gentry--Peikert--Vaikuntanathan~(GPV)~\cite{DBLP:conf/stoc/GentryPV08}
`hash-and-sign' framework for lattice-based digital signatures. It
uses the NTRU family of lattices, and its trapdoor sampler involves
several algorithmic innovations and optimizations to obtain good time
and space efficiency. It has been selected as one of three finalist
signature schemes in the NIST Post-Quantum Cryptography project.

\subsection{Re-Signing Strategies}
\label{sec:re-signing}

In the GPV framework, each message to be signed is first hashed to a
digest `syndrome' (more formally, a coset of a certain lattice defined
by the public key). Then, a `trapdoor sampler' algorithm is used to
randomly select one of many valid signatures for this syndrome. For
security, the following rule must be maintained:
\begin{center}
  \emph{\textbf{for any given public key,\\the signer must never
      reveal two inequivalent signatures for the same digest
      syndrome}.}
\end{center}
`Inequivalent' means that the signatures represent different short
vectors having the same syndrome; signatures representing the same
short vector under different valid encodings are equivalent.

Violating the above rule would break a central assumption needed by
the GPV unforgeability proof. Moreover, it may even enable an attacker
to forge signatures, because it would reveal short vectors in the
lattice associated with the public key. GPV recalls three generic
options for enforcing the rule:
\begin{itemize}
\item \textbf{Statefulness.} The signer can maintain a list of the
  signed messages and their signatures, and return the same signature
  if the same message is ever to be re-signed. This approach is very
  difficult to properly implement in practice, especially (but not
  only) if multiple devices may need to sign messages using the same
  private key.

\item \textbf{Randomized hashing.} Whenever signing a message, the
  signer can choose a random `salt' (or `nonce') and incorporate it
  into the hashing process to obtain a randomized digest. (The salt is
  included with the signature so that the verifier can compute the
  same digest.) If the salt has sufficient randomness, then no salt
  value will ever be repeated, so neither will any digest syndrome
  (under typical assumptions about hash function). Importantly, this
  non-repetition property is all that is required by the GPV security
  proof in the random-oracle model; one could even allow the adversary
  to choose the salt, as long as no message-salt pair is ever
  repeated.

\item \textbf{Derandomization.} When signing a message, the signer can
  use a deterministic hash function to map the message to a syndrome,
  and use a pseudorandom function (applied to the message) as the
  source of the random choices for the trapdoor sampler. So, when
  re-signing the same message, the same pseudorandom choices are made,
  yielding the same ultimate output.

  However, for this strategy to work in practice, it is very important
  that the entire sampling procedure be \emph{functionally equivalent}
  across all the relevant signing devices, where `device' broadly
  encompasses the entire relevant computing stack: hardware
  processors, operating system, compiler (including versions and
  optimizations), implementation of the sampling algorithm, etc.

  A further complication for Falcon is the fact that its trapdoor
  sampler inherently relies on \emph{floating-point}~(FP)
  operations. It is well known that slight deviations in FP
  computations can arise due to optimizations like reordered or
  combined instructions, or even slight differences in the
  implementations of floating-point units (FPUs).
\end{itemize}

The Falcon specification opts to use hash randomization, using a
$40$-byte ($320$-bit) salt value that is prepended to the message
before hashing (and included in the signature). This approach provides
the flexibility to safely use specialized optimizations, which may
yield functionally inequivalent signing procedures, across a variety
of computing devices and configurations. However, it also requires
signing devices to have a high-quality source of randomness, which is
not available in all contexts.

\subsection{SNARK (Un)Friendliness}
\label{sec:snark-unfriendliness}

Hash randomization means that the digest syndrome depends on the
random salt in the signature. This syndrome is computed using a real
hash function that is used to instantiate an idealized random
oracle. For recommended hash functions like SHAKE, this makes
signature verification very `unfriendly' for SNARKs.\footnote{A SNARK
  is a Succinct Noninteractive ARgument of Knowledge; it allows one to
  prove complex NP statements, possibly having large witnesses, via
  short proofs that are often also very fast to verify.}

More concretely: for a given message, to prove knowledge of a valid
Falcon signature for this message (relative to some public key), the
SNARK would need to embed the full computation of the digest syndrome,
because it depends on the salt in the
signature.\footnote{Alternatively, once could keep the salt `in the
  clear' and compute the syndrome digest `outside the SNARK.' However,
  this requires that the SNARK verifier know the salt (along with the
  message), and it is not succinct in the multiple-signature context
  described next, because all the various salts would need to be known
  by the verifier.}  More generally, suppose one wishes to succinctly
prove knowledge of \emph{many} signatures (relative to respective
public keys) for the same message, as in compact
certificates~\cite{DBLP:conf/sp/MicaliRVWZ21}. Then the SNARK must
embed the computations of \emph{all} the digest syndromes, which vary
from signature to signature due to the different salts. For existing
SNARK systems, this would require a prohibitive amount of computation
for the SNARK prover.

\paragraph{SNARK-friendliness of derandomization.}

On the other hand, a \emph{derandomized} GPV signature scheme is much
more SNARK-friendly in the above scenarios. This is because the digest
syndrome depends \emph{only on the message}, which is known to the
verifier as (part of) the statement to be proved. So, the hashing can
be done `outside the SNARK', i.e., both prover and verifier can
compute the digest, and the SNARK can simply prove that the signature
is valid relative to that digest. For GPV signatures, this part of the
signature verification is very SNARK-friendly, consisting essentially
of just modular linear equations and enforcement of a norm bound
(essentially, a range check). More broadly, in the single-message,
many-signatures scenario of compact certificates, the \emph{same}
digest syndrome is used for every signature. So, this single hash
computation can again be done outside the SNARK, or even inside the
SNARK just once, and amortized over all the signatures.

\subsection{Deterministic Falcon}
\label{sec:deterministic-falcon}

Motivated by the above considerations, this document specifies a
derandomized---or more precisely, `deterministic signing'---variant of
Falcon, following the third option above. Despite the qualitative
differences between this approach and randomized hashing, it turns out
that it is straightforward to define a deterministic mode in terms of
the randomized mode via minor tweaks; see \cref{sec:specification}.
Moreover, the deterministic mode can be implemented easily and
robustly using Falcon's existing interface and implementation, with
just a small amount of `wrapper' code; see \cref{sec:implementation}.

% flatex input end: [overview.tex]

% list of notes to address; appears only in draft mode

% flatex input: [spec.tex]

\newcommand{\sfname}[1]{\ensuremath{\mathsf{#1}}\xspace}

\newcommand{\unsaltedsig}{\sfname{unsalted\textunderscore sig}}
\newcommand{\saltedsig}{\sfname{salted\textunderscore sig}}
\newcommand{\saltversionbyte}{\sfname{salt\textunderscore version\textunderscore byte}}
\newcommand{\saltedheaderbyte}{\sfname{salted\textunderscore header\textunderscore byte}}
\newcommand{\unsaltedheaderbyte}{\sfname{unsalted\textunderscore header\textunderscore byte}}
\newcommand{\rbytes}{\sfname{r\textunderscore bytes}}
\newcommand{\sbytes}{\sfname{s\textunderscore bytes}}
\newcommand{\preversionedsalt}{\sfname{preversioned\textunderscore salt}}

\section{Specification}
\label{sec:specification}

This section specifies a deterministic mode for Falcon. In short, this
mode essentially removes the salt from ordinary Falcon signatures,
instead using a fixed salt value (for compatibility), and the signer
uses deterministic pseudorandom choices in its sampling algorithm.

\subsection{Determinism and Versioned Salt}
\label{sec:versioned-salt}

In deterministic mode, the signer first deterministically maps the
message to a syndrome digest, by hashing (just as in the randomized
mode) the message and a certain \emph{fixed} salt value. It then
deterministically generates a signature for this digest, by supplying
a trapdoor sampling algorithm with its secret `random' choices using a
pseudorandom function of the message. The precise details of the
sampling procedure and the pseudorandom function are left unspecified
here, and are determined by the implementation; the goal is for a
particular family of implementations to instantiate the same
deterministic behavior. It is stressed that
\begin{center}
  \emph{the \textbf{same private key} should not be used to sign the
    \textbf{same message digest}\\using \textbf{functionally
      inequivalent} sampling procedures.}
\end{center}
Doing so may violate the desired determinism, and thereby the central
rule needed for security (see \cref{sec:re-signing}).

In particular, if the input-output functionality of the sampling
procedure ever changes---e.g., due to a bug fix, an optimization, or a
port to an different kind of computing device---then either the
private key or the deterministic mapping from messages to syndromes
should be completely refreshed.  One way to achieve this refreshing is
to generate and distribute entirely new keys for use with the new
procedure; however, this may be operationally difficult in certain
applications.

As an alternative to refreshing keys, this specification includes a
versioning mechanism that allows for refreshing the
message-to-syndrome map, by slightly changing the salt. More
specifically, each signature contains a `salt version,' which is
included as part of the otherwise fixed salt used by the signer and
verifier. When modeling the hash function as a random oracle (as is
standard with GPV), changing the salt version yields a completely
fresh random digest for every message. Therefore, signers can safely
use inequivalent sampling procedures with the same secret key, simply
by using different salt versions.\footnote{Note that for the SNARKed
  compact certificate application described in
  \cref{sec:snark-unfriendliness}, it is best if all the signatures in
  a given certificate use the same salt version, so that the SNARK
  needs to prove correctness of only a single digest. But the salt
  version may still change from one certificate and SNARK proof to
  another.}

\subsection{Keys}
\label{sec:keys}

In deterministic Falcon, public verification keys and private signing
keys are identical to those in randomized Falcon. Moreover, they
are safely interoperable between modes: a private key may be used to
sign messages in both randomized and deterministic mode, with no loss
in security versus using just one of these modes. This is simply
because the randomized mode is vanishingly unlikely to choose any of
the (versioned) salt values that the deterministic mode uses.

\subsection{Signatures}
\label{sec:signatures}

In deterministic Falcon, signatures have an `unsalted' format, which
is just a slight modification of the salted format for standard
(randomized) Falcon. In summary, unsalted format implicitly uses a
fixed (versioned) salt value, which is omitted from the signatures
themselves, and the header byte is slightly different to indicate the
different format.

\subsubsection{Salted and Unsalted Formats}
\label{sec:formats}

\paragraph{Salted signatures.}

First recall from~\cite[Section~3.11.3]{falcon} that in randomized
Falcon, a signature contains a `salt' (or `nonce') value, and has the
format
\[ \saltedsig = \saltedheaderbyte \| \rbytes \| \sbytes \; \text{,} \]
where
\begin{itemize}[itemsep=0pt]
\item \saltedheaderbyte is a header byte that defines the Falcon dimension
  parameter~$n=2^{\ell}$ and the signature encoding, as defined
  in~\cite[Section~3.11.3]{falcon}. Note that the most-significant bit
  of \saltedheaderbyte is always~$0$.
\item \rbytes is a $40$-byte salt string.
\item \sbytes is a string of bytes, whose length depends on~$n$ and
  the signature encoding, representing the core of the signature
  (i.e., the output of the trapdoor sampler).
\end{itemize}

\paragraph{Unsalted signatures.}

An unsalted signature \unsaltedsig essentially just replaces the
arbitrary salt string \rbytes with a single byte specifying a fixed
(versioned) salt string. Specifically, it has the format
\begin{equation}
  \label{eq:unsaltedsig}
  \unsaltedsig = \unsaltedheaderbyte \| \saltversionbyte \| \sbytes \;
  \text{,}
\end{equation}
where:
\begin{itemize}[itemsep=0pt]
\item \unsaltedheaderbyte is exactly as in salted Falcon signatures
  (see above), except that its most-significant bit is always~$1$
  (not~$0$).
\item $\saltversionbyte$ is a byte specifying the version of the fixed
  salt that was used by the signer. This implicitly selects a specific
  message-to-syndrome mapping, as describe at the start of the
  section.
\item \sbytes is exactly as in salted Falcon signatures.
\end{itemize}

\subsubsection{Validity and Verification}
\label{sec:sig-validity}

For a given public key and message, an unsalted signature \unsaltedsig
as in \cref{eq:unsaltedsig} is defined to be valid if and only if the
corresponding salted Falcon signature
\begin{equation}
  \label{eq:saltedsig}
  \saltedsig = \saltedheaderbyte \| \saltversionbyte \|
  \preversionedsalt_{\ell} \| \sbytes
\end{equation}
is valid for the same public key and message (respectively), where:
\begin{itemize}[itemsep=0pt]
\item \saltedheaderbyte is simply \unsaltedheaderbyte with its
  most-significant bit set to~$0$, and
\item $\preversionedsalt_{\ell}$ is a certain $39$-byte string that
  depends only on the parameter~$\ell=\log_{2}(n)$ defined by the
  header byte; this string is defined in \cref{sec:preversioned-salt}
  below.
\end{itemize}
Note that in \saltedsig, the $40$-byte string
$\rbytes := \saltversionbyte \| \preversionedsalt_{\ell}$ serves as
the salt.

Therefore, to verify \unsaltedsig for a given public key and message,
one can simply construct the corresponding \saltedsig and verify it,
as follows:
\begin{enumerate}[itemsep=0pt]
\item check that the most-significant bit of the header byte is~$1$,
  change it to~$0$, and determine the value of~$\ell$ defined by the
  header;
\item insert $\preversionedsalt_{\ell}$ between \saltversionbyte and \sbytes,
  and
\item verify the resulting \saltedsig as an ordinary salted Falcon
  signature, for the same public key and message (respectively).
\end{enumerate}

\subsubsection{Pre-Versioned Salt}
\label{sec:preversioned-salt}

For any legal value of $\ell=\log_{2}(n)$ for Falcon (i.e.,
$\ell \in \{1,2,\ldots,10\}$), the $39$-byte string
$\preversionedsalt_{\ell}$ is defined as:
\begin{enumerate}[itemsep=0pt]
\item a byte representing the value of~$\ell$ (e.g., \texttt{0x09} for
  $n=512$, and \texttt{0x0A} for $n=10$)\footnote{Due to the format of
    the header byte, the byte representing~$\ell$ can be obtained as
    $\unsaltedheaderbyte \text{ AND } \mathtt{0x0F}$.}, followed by
\item the ASCII representation of \texttt{FALCON\textunderscore DET},
  followed by
\item the required number of all-zero padding bytes to make the length
  exactly~$39$ bytes.
\end{enumerate}
(The string \texttt{FALCON\textunderscore DET} is included simply to
provide domain separation from other uses of the hash function.)

% flatex input end: [spec.tex]

% list of notes to address; appears only in draft mode

% flatex input: [impl.tex]

\newcommand{\fd}{\texttt{falcon\textunderscore det1024}\xspace}

\section{Implementation}
\label{sec:implementation}

This section describes an implementation, called
\fd~\cite{falcon_det1024}, of a \emph{subset} of the deterministic
Falcon signature scheme defined in \cref{sec:specification}. This
implementation is obtained by just a small amount of simple `wrapper'
code around the official, unchanged Falcon interface and
implementation.

The functionality implemented by \fd is limited to fixed Falcon
parameter $n=2^{10}=1024$, which targets NIST post-quantum security
category~$5$ (the highest defined level). It is also limited to
signatures in either of Falcon's `compressed' or `constant-time' (CT)
formats, which have variable and fixed lengths, respectively; it does
not support the fixed-length `padded' format. More specifically, all
functions are limited to $n=1024$, and:
\begin{itemize}[itemsep=0pt]
\item signing creates a signature \emph{in compressed format only};
\item there is an additional function that converts a given signature
  from compressed format to CT format;
\item verification accepts a signature if and only if it is in the
  caller-specified format---\emph{either compressed or CT}---and it is
  valid according to the specification from \cref{sec:signatures}.
\end{itemize}
The rationale for the choice of supported (and unsupported) signature
formats is given below in \cref{sec:rationale-formats}.

\paragraph{Strong unforgeability.}

Because \fd supports transcoding from one signature format to another,
it does not, strictly speaking, preserve Falcon's strong
unforgeability (if an application ever accepts signatures in CT
format).\footnote{Note that Falcon's signatures are inherently
  transcodable, so it is strongly unforgeable only if the application
  enforces the use of a single signature format when signing and
  verifying. However, the envisioned applications of \fd will violate
  this requirement.}  However, \fd does still preserve a slightly
relaxed form of strong unforgeability, where the set of
message-signature pairs revealed by the signer is defined to include
both the directly released compressed-form signatures \emph{and} their
transcoded CT-form versions.

\subsection{Keys}
\label{sec:impl-keys}

As mentioned in \cref{sec:keys}, randomized and deterministic Falcon
have identical keys, and keys are interoperable between the two
modes. Therefore, \fd simply invokes Falcon's key-generation function
on suitable fixed arguments corresponding to $n=1024$.

\subsection{Signature Verification}
\label{sec:impl-verification}

To verify a given deterministic signature, \fd checks that the
signature has the expected parameter $n=1024$ and caller-specified
format (either compressed or CT), then proceeds according to the steps
given in \cref{sec:sig-validity}: it constructs the corresponding
salted signature, and then verifies it using Falcon's verification
procedure as a `black box.'

\subsection{Signature Conversion}
\label{sec:signature-conversion}

To convert a signature from compressed format to CT format, \fd simply
uses Falcon's internal encoding and decoding functions to decompress
the given signature and then re-encode it. This transcoding does not
require any secret information, nor even any other public information
apart from the signature itself.

\subsection{Robust Deterministic Signing}
\label{sec:robust-determ-sign}

The signing procedure in \fd is more subtle than the other
procedures. However, it is still a straightforward application of
Falcon's `streamed' interface, which:
\begin{itemize}[itemsep=0pt]
\item externalizes (to the caller) the hashing of the salt and the
  data to be signed\footnote{For consistency with the Falcon
    implementation, in this section, the term `data' is used for the
    message to be signed.}, and
\item allows the client to provide (the state of) a pseudorandom
  generator, which the sampling procedure uses for its secret random
  choices.
\end{itemize}
This interface allows \fd to hash the data with a fixed (versioned)
salt, and to provide a deterministic stream of secret pseudorandom
bits that depends solely on the private key and the data to be
signed. Hence, when the same data is re-signed under the same private
key, the same stream is produced, thus yielding a fully deterministic
signing procedure across all conforming computing devices (though see
\cref{sec:robustness} below for caveats).

\subsubsection{Implementation Details}
\label{sec:sign-impl-details}

Falcon's streamed interface provides a function
\texttt{falcon\textunderscore sign\textunderscore dyn\textunderscore
  finish}, which produces a salted signature when given the following
arguments (among others that are not relevant to the present
discussion):
\begin{itemize}[itemsep=0pt]
\item a SHAKE context \texttt{hash\textunderscore data} that
  represents the hashed salt and data, and
\item another SHAKE context~\texttt{rng} that is used to generate a
  stream of secret pseudorandom bytes for the trapdoor-sampling
  procedure.
\end{itemize}

The \fd signing procedure calls \texttt{falcon\textunderscore
  sign\textunderscore dyn\textunderscore finish} with the following
SHAKE contexts to obtain a salted signature with a fixed (versioned)
salt string, then removes the salt (except for the version byte) and
tweaks the header byte to produce the corresponding unsalted
signature.
\begin{itemize}[itemsep=0pt]
\item The \texttt{hash\textunderscore data} context is prepared by
  injecting the appropriate (versioned) salt string, then the data to
  be signed.
\item The \texttt{rng} context is prepared by injecting a byte
  representing the $\mathtt{logn}=10$ parameter, then the entire
  private key string \texttt{privkey}, then the \texttt{data} to be
  signed. The output of this~\texttt{rng} context represents the
  function
  \[ F_{\mathtt{privkey}}(\mathtt{data}) :=
    \mathtt{SHAKE}(\mathtt{logn} \| \mathtt{privkey} \| \mathtt{data})
    \; \text{.}\footnote{The \texttt{logn} byte is used for domain
      separation.} \] Under standard assumptions about SHAKE, for any
  fixed byte value of~\texttt{logn}, this is a deterministic,
  pseudorandom function of just the private key and the data, as
  desired.
\end{itemize}
Note that both of the above SHAKE contexts require the injection of
the entire data to be signed; for extremely long data, this may result
in a noticeable slowdown. In such a case, applications may opt to
`pre-hash' the data using a collision-resistant hash function and
instead sign the hash value.\footnote{Alternatively, in just the
  \texttt{rng} context, the \texttt{data} input could be replaced by a
  short collision-resistant hash. However, this change would yield a
  highly inequivalent signing procedure, and in particular would be
  incompatible with the known-answer tests for the deterministic
  mode.}

\subsubsection{Robustness Across Devices}
\label{sec:robustness}

For fully deterministic signing across different computing devices, it
is not enough to generate a repeatable stream of pseudorandom bytes;
the actual signing procedures that \emph{consume} those bytes should
ideally be \emph{functionally equivalent}, i.e., they should have the
same input-output behavior. Failing that, the procedures should be
near-equivalent, i.e., equivalent on `almost all' inputs, keeping in
mind that the attacker may have partial or total control over the data
to be signed.

For Falcon, ensuring functional (near-)equivalence can be a delicate
matter, because the signing procedure internally uses a sophisticated
algorithm that relies on floating-point operations. The presence of
different floating-point units and code optimizations, such as `fused
multiply-and-add'~(FMA), on different devices can potentially result
in slight discrepancies that could result in functional inequivalence.

\paragraph{Floating-point emulation.}

Fortunately, Falcon includes a floating-point emulation mode
(\texttt{FALCON\textunderscore FPEMU}), which implements the required
subset of floating-point operations using $32$- or $64$-bit
integers. This mode is fully deterministic and constant time, assuming
a C99-compliant compiler and hardware that implements a certain few
operations in constant time. On modern desktop/server-class CPUs, the
emulation slows down key generation by only about a 2x factor,
signature generation by about a 15x factor, and signature verification
not at all (because it does not use floating-point operations).

By default, \fd uses Falcon's floating-point emulation, and also
explicitly disables certain optimizations that could potentially yield
functional inequivalence.\footnote{The Falcon implementation already
  disables most of these optimizations when floating-point emulation
  is enabled, but \fd disables them explicitly as a defensive
  measure.} It is \emph{strongly recommended} that applications retain
this emulation, unless performance considerations absolutely require
otherwise. In such a case, caution should be exercised to ensure
functional (near-)equivalence, and certainly compliance with the test
vectors (see below). Where possible, operational restrictions may also
be appropriate, such as restricting the use of every private key to a
specific kind of device and configuration.

\paragraph{Test vectors.}

To help `sanity check' other devices, implementations, compilers,
optimizations, etc.\ for functional (near-)equivalence, \fd includes
known-answer tests~(KATs) for its deterministic mode, along with a
\texttt{test\textunderscore deterministic} program for checking them
(and generating more, if desired). Any deviation from the KATs
indicates a lack of the desired equivalence. However, agreement with
all the KATs does not prove equivalence for all possible inputs; for
this, one could potentially use an automated theorem prover.

The authors have checked the KATs on modern Intel CPUs for a variety
of Falcon options (native versus emulated floating-point operations,
AVX2 and FMA optimizations), compilers (\texttt{gcc} and
\texttt{clang}), and compiler optimizations (\texttt{-O0} through
\texttt{-O3}). Under all these variants, \emph{no deviation from the
  KATs has been detected}. While this does not come close to proving
that all the various procedures are equivalent, it does show that the
Falcon code is not very sensitive to configuration changes on the
available computing platforms of interest.

\subsection{Rationale for Supported Signature Formats}
\label{sec:rationale-formats}

The rationale for the supported signature formats---signing only in
compressed format, verifying in either compressed or CT format, and
not allowing padded format at all---is as follows.

\paragraph{Signing in only one format.}

Signing is limited to just one format (i.e., compressed) in order to
robustly adhere to the central rule needed for security: the signer
must never reveal two \emph{inequivalent} signatures for the same
digest syndrome (see~\cref{sec:re-signing}). Allowing the signer to
output signatures in two different formats might open the possibility
of violating this rule.

Note that the current official Falcon implementation actually
\emph{can} sign deterministically in both compressed and CT formats
without violating the above rule, because its signing procedure
generates \emph{equivalent} signatures in these formats, when all the
inputs (other than the requested format) remain the same. However,
this is \emph{not} the case for `padded' format, because Falcon's
signer retries whenever a candidate signature does not fit into padded
format, even though it may fit into compressed and CT formats. Given
these subtleties, the possibility of a future change in
implementation, and the low cost of transforming between formats, it
seems best to restrict deterministic signing to just one format.

\paragraph{Properties of the formats.}

Compressed format, which has a variable length, is Falcon's default,
and has the shortest average length of all of the formats. Padded
format has a fixed length, which is only slightly longer than the
average compressed-format length. Clearly, shorter signatures reduce
the communication required for sending many signatures over the
network to prepare compact certificates, and the size of such
certificates. However, compressed and padded formats are quite
SNARK-unfriendly, because they require a complicated decompression
algorithm that has many data-dependent conditionals and loops.

On the other side of the spectrum, CT format is significantly larger
(about 25\% more than compressed and padded formats), so using it for
network communication and in storage is wasteful. However, it is very
SNARK-friendly, because it represents each signed-integer polynomial
coefficient with a fixed number of bits, in two's
complement. Therefore, each coefficient can be reconstructed via a
fixed linear function of the corresponding block of bits, which is
very inexpensive in terms of SNARK constraints.

\paragraph{Interchangeability.}

CT format is \emph{fully interchangeable} (in both directions) with
compressed format, but is only interchangeable with padded format in
one direction.

In more detail: any valid, possibly maliciously generated
compressed-format signature (for any given public key and message,
which may also be maliciously generated) can be efficiently
transformed into a valid CT-format signature (for the same public key
and message), and vice-versa. The latter direction works because the
variable length of compressed format can expand to represent any valid
signature, with a proven maximum length. By contrast, any valid
padded-format signature can be efficiently transformed into a valid
CT-format signature, \emph{but not necessarily vice-versa}. This is
because transforming a valid CT-format signature may require more
bytes than are allowed by padded format.

\paragraph{SNARKs.}

Full interchangeability is very important for the correctness of
transforming a compact certificate into a SNARK proof, and also for
the security argument connecting the soundness of the SNARK to the
security of compact certificates, and then to the underlying signature
scheme. More specifically:
\begin{itemize}
\item Any valid compact certificate---i.e., one that satisfies its
  verifier---must be efficiently transformable to a valid SNARK
  proof---i.e., one that satisfies its own verifier. As mentioned
  above, the SNARKed circuit needs signatures to be given in CT
  format, but compact certificates (and the network used to create
  them) may wish to use a more compact format. Therefore, any valid
  signature in the compact format must always be efficiently
  transformable into a valid signature in CT format. Fortunately, this
  is the case for both of Falcon's `compressed' and `padded' formats.
  
\item The security argument uses the opposite direction: it requires
  that any witness that satisfies the SNARKed circuit---i.e., one
  containing valid CT-format signatures---can be efficiently
  transformed into a valid compact certificate (or a `break' of the
  SNARK or related cryptographic primitive). In addition, any valid
  compact certificate should be transformable into one or more valid
  signatures (or a `break' of a component primitive), etc.
\end{itemize}

Because all Falcon signature formats can be transformed to CT format,
one way to achieve the above goals is to ensure that CT format is
accepted by the compact-certificate verifier \emph{and any other
  subcomponent that verifies Falcon signatures}. However, this is a
brittle solution, because it requires every verifier in the chain to
accept a format that may be unnatural for its specific purpose, and
may even be unused in reality.

A robust approach, which is what \fd adopts, is simply to ensure that
all signature formats that may be accepted by the various verifiers
(the SNARK circuit, the compact-certificate verifier, etc.) are fully
interchangeable. This ensures that valid witnesses can always be
transformed to include signatures of the desired format(s). This
full-interchangeability policy leads \fd to support compressed and CT
formats, but not padded format.

% flatex input end: [impl.tex]

% list of notes to address; appears only in draft mode

%FLATEX-REM:\bibliography{falcon}
%*flatex input: [falcon-det.bbl]
\newcommand{\etalchar}[1]{$^{#1}$}
\begin{thebibliography}{MRV{\etalchar{+}}21}
\expandafter\ifx\csname urlstyle\endcsname\relax
  \providecommand{\doi}[1]{doi:\discretionary{}{}{}#1}\else
  \providecommand{\doi}{doi:\discretionary{}{}{}\begingroup
  \urlstyle{rm}\Url}\fi

\bibitem[FHK{\etalchar{+}}20]{falcon}
P.-A. Fouque, J.~Hoffstein, P.~Kirchner, V.~Lyubashevsky, T.~Pornin, T.~Prest,
  T.~Ricosset, G.~Seiler, W.~Whyte, and Z.~Zhang.
\newblock Falcon: Fast-{Fourier} lattice-based compact signatures over {NTRU},
  2020.
\newblock Specification v1.2, \url{https://falcon-sign.info/}.

\bibitem[GPV08]{DBLP:conf/stoc/GentryPV08}
C.~Gentry, C.~Peikert, and V.~Vaikuntanathan.
\newblock Trapdoors for hard lattices and new cryptographic constructions.
\newblock In \emph{STOC}, pages 197--206. 2008.

\bibitem[LP21]{falcon_det1024}
D.~Lazar and C.~Peikert.
\newblock Deterministic {Falcon}-1024 implementation, November 2021.
\newblock \url{https://github.com/Algorand/falcon}.

\bibitem[MRV{\etalchar{+}}21]{DBLP:conf/sp/MicaliRVWZ21}
S.~Micali, L.~Reyzin, G.~Vlachos, R.~S. Wahby, and N.~Zeldovich.
\newblock Compact certificates of collective knowledge.
\newblock In \emph{{IEEE} Symposium on Security and Privacy}, pages 626--641.
  2021.

\end{thebibliography}

% flatex input end: [falcon-det.bbl]
%FLATEX-REM:\bibliographystyle{alphaabbrvprelim}

\end{document}


%%% Local Variables:
%%% mode: latex
%%% TeX-master: t
%%% End:

% flatex input end: [falcon-det.tex]
