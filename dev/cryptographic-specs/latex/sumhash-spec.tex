%% WARNING: THIS IS A DERIVED (FLATTENED) FILE; DO NOT MAKE
%% SUBSTANTIVE EDITS! EDIT THE ORIGINAL SOURCE FILES INSTEAD.

% flatex input: [sumhash-spec.tex]
\documentclass[11pt]{article}
\usepackage{fullpage}
\usepackage{microtype}
\usepackage{newtxtext}
\usepackage{enumitem}
\usepackage{xspace}

\usepackage{amssymb,amsmath,amsfonts}
\usepackage[amsmath,amsthm,thmmarks,hyperref]{ntheorem}
\usepackage{mathtools}
\usepackage{hyperref}
\usepackage[capitalize,nameinlink,noabbrev]{cleveref} % load after hyperref

\usepackage[normalem]{ulem}
\usepackage{url}
\usepackage{breakcites}
\usepackage{authblk}

% remove 'draft' to turn off fixme notes
\usepackage[multiuser,inline,nomargin]{fixme}

% fixme: register commands for author(s)
\FXRegisterAuthor{c}{ec}{\color{red}Chris}
\fxusetheme{color}

% manual import: [head.tex], recursively, removing unused macros

% blackboard symbols

\newcommand{\N}{\ensuremath{\mathbb{N}}}
\newcommand{\Z}{\ensuremath{\mathbb{Z}}}

\newcommand{\Zq}{\ensuremath{\Z_q}}

% "left-right" pairs of symbols

%% NOTE: this requires \usepackage{mathtools} in the document preamble

% inner product
\DeclarePairedDelimiter\inner{\langle}{\rangle}
% absolute value
\DeclarePairedDelimiter\abs{\lvert}{\rvert}
% a set
\DeclarePairedDelimiter\set{\{}{\}}
% parens
\DeclarePairedDelimiter\parens{(}{)}
% floor function
\DeclarePairedDelimiter\floor{\lfloor}{\rfloor}
% ceiling function
\DeclarePairedDelimiter\ceil{\lceil}{\rceil}

% macros for matrices and vectors
\newcommand{\matA}{\ensuremath{\mathbf{A}}}

\newcommand{\veca}{\ensuremath{\mathbf{a}}}
\newcommand{\vecx}{\ensuremath{\mathbf{x}}}
\newcommand{\vecy}{\ensuremath{\mathbf{y}}}
\newcommand{\vecz}{\ensuremath{\mathbf{z}}}
\newcommand{\veczero}{\ensuremath{\mathbf{0}}}
\newcommand{\vecone}{\ensuremath{\mathbf{1}}}

% equation numbering style
\numberwithin{equation}{section}

% GENERAL COMPUTING STUFF

\newcommand{\bit}{\ensuremath{\set{0,1}}}

\newcommand{\etal}{\textit{et al.}}

\renewcommand{\cal}[1]{\mathcal{#1}}

% show equation numbers only for referenced equations
% \mathtoolsset{showonlyrefs}

% flatex input: [macros.tex]
% additional project-specific macros
\DeclareMathOperator{\pad}{pad}

% flatex input end: [macros.tex]

% fixme: register commands for author(s)

\begin{document}

\title{Subset-Sum Hash Specification}

% date of last substantive change
\date{September 3, 2021}

\author{Yossi Gilad}
\author{David Lazar}
\author{Chris Peikert}
\affil{Algorand, Inc.}

\maketitle

% list of notes to address; appears only in draft mode
\listoffixmes

% flatex input: [notation.tex]

\section{Notation and Background}
\label{sec:notation-background}

All logarithms are base two, i.e., $\log = \log_{2}$, unless otherwise
indicated by a different subscript.

\paragraph{Binary strings.}

For a positive integer~$L$, $\bit^{< L}$ denotes the set of binary
strings of length strictly less than~$L$ (including the zero-length
empty string~$\varepsilon$), and $\bit^{*}$ denotes the set of all
binary strings of any length. For binary strings $u,v$ of any length,
$u v$ denotes their concatenation. For a binary string
$x \in \bit^{*}$, $\abs{x} \geq 0$ denotes its length in bits. For a
positive integer~$e$ and a non-negative integer~$z < 2^{e}$,
$\inner{z}_{e} \in \bit^{e}$ denotes the little-endian representation
of~$z$ as a binary string of exactly~$e$ bits.

\paragraph{Modular integers.}

For a positive integer~$q$, $\Zq := \Z/q\Z$ denotes the (abelian) ring
of integers modulo~$q$. Formally, $\Zq$ is the set of \emph{cosets} of
the form
\[ c = \tilde{c} + q\Z := \set{\ldots, \tilde{c}-2q, \tilde{c}-q,
    \tilde{c}, \tilde{c}+q, \tilde{c}+2q, \ldots} \] for some integer
$\tilde{c} \in \Z$, with $\tilde{c} + q\Z = \tilde{c}' + q\Z$ if and
only if~$q$ divides $\tilde{c}-\tilde{c}'$. In general, a coset can be
represented by any of its elements, which might even differ from one
location to the next. To avoid any ambiguity, this specification
always requires a coset $c \in \Zq$ to be externally read and written
using its (unique) \emph{distinguished representative}
$\bar{c} \in \set{0, 1, \ldots, q-1} \cap c$.

\paragraph{Vectors and matrices.}

Vectors are denoted by lower-case bold letters (e.g., $\veca$), and
matrices by upper-case bold letters (e.g.,~$\matA$). \emph{In this
  specification, vector and matrix entries are always indexed starting
  from zero.} A vector's $i$th entry is denoted by the same lower-case
letter, but without boldface, and with a subscript~$i$; e.g., $x_{i}$
is the $i$the entry of~$\vecx$. Similarly, the $i$th column of a
matrix is denoted by the same letter, but in lower case, and with a
subscript~$i$; e.g.,~$\veca_{i}$ is the $i$th column of~$\matA$.

The set of $n$-dimensional vectors over a set~$X$ is
denoted~$X^{n}$. In particular, $\Zq^{n}$ is an (abelian) additive
group, where the group operation is coordinate-wise addition
(modulo~$q$). Similarly, $X^{n \times m}$ denotes the set of
$n$-by-$m$-dimensional matrices over~$X$. For convenience, this
specification sometimes uses standard vector and matrix operations
(like sums and products) where they are well defined.


% flatex input end: [notation.tex]

% list of notes to address; appears only in draft mode

% flatex input: [function.tex]

\section{Compression Function Family}
\label{sec:function}

This section gives the mathematical definition of the subset-sum
compression function family, which was first studied and popularized
in early works
like~\cite{DBLP:journals/joc/ImpagliazzoN96,ajtai04:_gener_hard_instan_lattic_probl}. For
specific parameters, the concrete security of the function against
various kinds of (classical and quantum) cryptanalytic attacks is
analyzed in a separate work.

\subsection{Parameters}
\label{sec:parameters}

The subset-sum compression function family is parameterized by:
\begin{itemize}
\item a positive integer \emph{modulus} $q \in \N$ (often taken to be
  a power of two);
\item a positive integer \emph{dimension} $n \in \N$;
\item a positive integer \emph{input length} $m \in \N$, where $m > n
  \log q$.
\end{itemize}

\subsection{Function Definition}
\label{sec:function-definition}

The compression function family for parameters $q,n,m$ is defined as
the collection
\[ \cal F_{q,n,m} := \set{ f_{\matA} \colon \bit^{m} \to \Zq^{n} :
    \matA \in \Zq^{n \times m} } \text{,} \] where each
function~$f_{\matA} \in \cal F_{q,n,m}$ is defined as
\begin{equation}
  \label{eq:f_A}
  f_{\matA}(\vecx) := \matA \vecx = \sum_{i=1}^{m} x_{i} \cdot
  \veca_{i} = \sum_{i : x_{i}=1} \veca_{i} \in \Zq^{n} \text{.}
\end{equation}
The latter summation explains the name ``subset-sum hash'': the output
is the subset-sum of the columns of~$\matA$ indicated by the bits of
the input~$\vecx$.

Observe that the condition $m > n \log q$ ensures that the functions
in the family are \emph{compressing}, i.e., the cardinality of their
common domain~$\bit^{m}$ is strictly larger than that of their common
range~$\Zq^{n}$: $2^{m} > 2^{n \log q} = q^{n}$.

When $q = 2^{u}$ is a power of two, $\ell := n \log q = n u$ is called
the \emph{output length}, and $b := m - \ell > 0$ is called the
\emph{block length}. In this case, for any $\vecy \in \Zq^{n}$, let
$\inner{\vecy} \in \bit^{\ell}$ denote the representation of~$\vecy$
as an $\ell$-bit string, obtained as the concatenation of the
($u$-bit, little-endian representations of the) distinguished
representatives $\bar{y}_{i} \in \set{0, 1, \ldots, q-1}$ of the
coordinates~$y_{i}$:
\begin{equation}
  \label{eq:Zqn-rep}
  \inner{\vecy} := \inner{\bar{y}_{0}}_{u} \inner{\bar{y}_{1}}_{u}
  \cdots \inner{\bar{y}_{n-1}}_{u} \in \bit^{\ell} \text{.}
\end{equation}

\subsection{Security Properties}
\label{sec:security-properties}

\paragraph{Conjectured properties.}

For appropriate parameters, the subset-sum compression function family
$\cal F_{q,n,m}$ is conjectured to have the following security
properties:
\begin{itemize}
\item \emph{Uninvertibility (UI):} given uniformly random and
  independent $\matA \in \Zq^{n \times m}$ and $\vecy \in \Zq^{n}$, it
  is infeasible to find some $\vecx \in \bit^{m}$ such that
  $f_{\matA}(\vecx) = \matA \vecx = \vecy$.

\item \emph{One-wayness (OW):} given $\matA \in \Zq^{n \times m}$ and
  $\vecy = f_{\matA}(\vecx) = \matA \vecx \in \Zq^{n}$ (but
  not~$\vecx$ itself), where $\matA \in \Zq^{n \times m}$ and
  $\vecx \in \bit^{m}$ are uniformly random and independent, it is
  infeasible to find some $\vecx' \in \bit^{m}$ (not necessarily
  different from~$\vecx$) such that $f_{\matA}(\vecx') = \vecy$.

\item \emph{Second-preimage resistance (SPR):} given uniformly random
  and independent $\matA \in \Zq^{n \times m}$ and
  $\vecx \in \bit^{m}$, it is infeasible to find some
  $\vecx' \in \bit^{m}$ such that $\vecx' \neq \vecx$ and
  $f_{\matA}(\vecx') = f_{\matA}(\vecx)$.

\item \emph{Target-collision resistance (TCR):} it is infeasible to
  choose some $\vecx \in \bit^{m}$ and then, given a uniformly random
  $\matA \in \Zq^{n \times m}$, to find some distinct
  $\vecx' \in \bit^{m} \setminus \set{\vecx}$ such that
  $f_{\matA}(\vecx) = f_{\matA}(\vecx')$, i.e.,
  $\matA \vecx = \matA \vecx' \in \Zq^{n}$.

\item \emph{Collision resistance (CR):} given a uniformly random
  $\matA \in \Zq^{n \times m}$, it is infeasible to find distinct
  $\vecx, \vecx' \in \bit^{m}$ such that
  $f_{\matA}(\vecx) = f_{\matA}(\vecx')$, i.e.,
  $\matA \vecx = \matA \vecx' \in \Zq^{n}$.

  Note that breaking CR is equivalent to finding a nonzero
  $\vecz \in \set{-1, 0, 1}^{m} \setminus \set{\veczero}$ such that
  $\matA \vecz = \veczero \in \Zq^{n}$. In one direction, given such
  $\vecx, \vecx'$, define
  $\vecz = \vecx - \vecx' \in \set{-1, 0, 1}^{m} \setminus
  \set{\veczero}$ and observe that
  $\matA \vecz = \matA \vecx - \matA \vecx' = \veczero$. In the other
  direction, given such~$\vecz$, let $\vecx \in \bit^{m}$ be~$1$
  (respectively,~$0$) wherever~$\vecz$ is~$1$ (resp.,~$0$ or $-1$),
  and similarly let $\vecx' \in \bit^{m}$ be~$1$ (respectively,~$0$)
  wherever~$\vecz$ is~$-1$ (resp.,~$0$ or~$1$). Then
  $\vecz = \vecx - \vecx'$, and since
  $\veczero = \matA \vecz = \matA(\vecx - \vecx')$, we have
  $\matA \vecx = \matA \vecx'$, as desired.
\end{itemize}
It is well known that CR tightly implies TCR generically (i.e., for
any compression function family), because breaking TCR immediately
yields a collision. Moreover, TCR tightly implies SPR generically,
because being able to find a collision with a uniformly random input
(that is independent of~$\matA$) in particular means being able to
find a collision with an input of one's choice (since that input may
just be chosen uniformly). Furthermore, for parameters that yield
significant compression (i.e., for $m \gg n \log q$), it is well known
that SPR tightly implies OW generically, and that OW implies UI for
the subset-sum family.

While none of the converse directions is known to hold generically,
for the subset-sum family with significant compression, a
\emph{relaxed} form of UI implies CR, thus implying that all four
security properties are closely related.\@ Specifically, any attack
that breaks CR with probability~$\delta$ can be used to successfully
attack relaxed-UI in essentially the same amount of time and with
probability $\approx \delta/(2m)$, where in relaxed-UI the
output~$\vecx$ may be taken from $\set{-1,0,1}^{m}$ (it is not limited
to $\bit^{m}$).

\paragraph{Non-conjectured properties.}

On the other hand, the family $\cal F_{q,n,m}$ is \emph{not
  conjectured to have}, or is even \emph{known not to have}, the
following security properties:
\begin{itemize}
\item \emph{Unpredictability/Pseudorandomness:} outputs of~$f_{\matA}$
  on different inputs do not appear random or uncorrelated, even if
  parts of the corresponding inputs are unknown to the attacker. This
  is because~$f_{\matA}$ is linear:
  \[ f_{\matA}(\vecx + \vecx') = \matA(\vecx + \vecx') = \matA \vecx +
    \matA \vecx' = f_{\matA}(\vecx) + f_{\matA}(\vecx') \text{,} \] as
  long as $\vecx, \vecx', \vecx+\vecx' \in \bit^{m}$, which is easy to
  arrange in many contexts. In particular, any~$0$ bits of~$\vecx$ can
  be changed to~$1$s by adding an~$\vecx'$ that is~$1$ in the suitable
  position(s), and~$0$ wherever~$\vecx$ is~$1$.

\item \emph{Random oracle:} for the same reasons as above, $f_{\matA}$
  does not ``behave like a random oracle'' in many of the ways that
  are typically expected. Therefore, it should not be used to
  instantiate a random oracle in any cryptosystems or protocols.
\end{itemize}

% flatex input end: [function.tex]

% list of notes to address; appears only in draft mode

% flatex input: [arbitrary.tex]

\section{Hashing Arbitrary-Length Messages}
\label{sec:arbitrary}

The compression functions defined in \cref{sec:function} map a
\emph{fixed-length} $m$-bit input to an output of length
$\ell := n \log q < m$. As is standard, a message of \emph{arbitrary}
length is hashed to a fixed-length output by invoking the compression
function one or more times, using the Merkle--Damg{\aa}rd (MD)
transform~\cite{DBLP:conf/crypto/Merkle89,DBLP:conf/crypto/Damgard89a}.

\subsection{Padding}
\label{sec:padding}

The transform uses the following padding method, which maps a binary
string of any bounded length into a strictly longer one whose length
is a multiple of the block length $b := m - \ell > 0$. Formally, for
any positive integer~$e$, define the padding function
$\pad_{b,e} \colon \bit^{< 2^{e}} \to \bit^{*}$ as
\begin{equation}
  \label{eq:pad}
  \pad_{b,e}(x) = x 1 0^{z} \inner{\abs{x}}_{e} \text{,}
\end{equation}
where $z \geq 0$ is the smallest non-negative integer for which
$\abs{x}+1+z+e$ is a multiple of~$b$. That is,
$r := b \cdot \ceil{k/b} - k \in \set{0, 1, \ldots, b-1}$, where
$k=\abs{x}+1+e$. Note especially that $\pad_{b,e}(x)$
\emph{unconditionally} appends at least $1+e$ bits to~$x$, even
if~$\abs{x}$ itself is a multiple of~$b$.

The purpose of the padding function is to produce a string whose
length is a multiple of the block length, and so that the MD transform
preserves collision resistance. That is, any collision in the full
hash function immediately yields a collision in the underlying
compression function.

\subsection{Hash Functions}
\label{sec:hash-functions}

Fix subset-sum parameters $q, n, m$ where $q=2^{u}$ is a power of two
for some positive integer~$u$.  Let $\ell := n \log q = n u$ be the
output length in bits, and let $b := m - \ell > 0$ be the block
length.

This specification defines two functions on arbitrary-length input
strings: an ``unsalted'' mode that takes only the string as input, and
a ``salted'' mode, originally defined
in~\cite{DBLP:conf/crypto/HaleviK06}, that additionally takes a public
salt value (which should typically be chosen uniformly at random, or
pseudorandomly).

\subsubsection{Unsalted Mode}
\label{sec:unsalted-mode}

For any matrix $\matA \in \Zq^{n \times m}$ defining the compression
function $f_{\matA} \colon \bit^{m} \to \Zq^{n}$ (\cref{sec:function})
and a positive integer $e > 0$, define the unsalted hash function
\begin{align}
  \label{eq:H}
  H_{\matA,e} &\colon \bit^{< 2^{e}} \to \bit^{\ell} \nonumber \\
  H_{\matA,e}(x) &:= H'_{\matA}(0^{\ell}, \pad_{b,e}(x)) \text{,}
\end{align}
where the chaining function
$H'_{\matA} \colon \bit^{\ell} \times \parens*{\bigcup_{i=0}^{\infty}
  \bit^{ib}} \to \bit^{\ell}$ is defined as
\begin{align}
  H'_{\matA}(h, w) &:= \begin{cases}
    h & \text{if } w = \varepsilon \\
    H'_{\matA}(\inner{f_{\matA}(h u)}, v) & \text{where } w = u v
    \text{ for } u \in \bit^{b} \text{.}
  \end{cases}
\end{align}
(Recall that representation $\inner{\vecy} \in \bit^{\ell}$ for
$\vecy \in \Zq^{n}$ is defined in \cref{sec:function-definition}.)

\subsubsection{Salted Mode}
\label{sec:salted-mode}

For any matrix $\matA \in \Zq^{n \times m}$ defining the compression
function $f_{\matA} \colon \bit^{m} \to \Zq^{n}$ (\cref{sec:function})
and a positive integer $e > 0$, define the salted hash function
\begin{align}
  \label{eq:H-tilde}
  \tilde{H}_{\matA,e} &\colon \bit^{< 2^{e}} \uwave{\times \bit^{b}} \to
                        \bit^{\ell} \nonumber \\
  \tilde{H}_{\matA,e}(x, \uwave{r}) &:= \tilde{H}'_{\matA}(0^{\ell},
                                     \pad_{b,e}(\uwave{0^{b}} x), \uwave{r})
\end{align}
where the chaining function
$\tilde{H}'_{\matA} \colon \bit^{\ell} \times
\parens*{\bigcup_{i=0}^{\infty} \bit^{ib}} \uwave{\times \bit^{b}} \to
\bit^{\ell}$ is defined as
\begin{align}
  \tilde{H}'_{\matA}(h, w, \uwave{r}) &:= \begin{cases}
    h & \text{if } w = \varepsilon \\
    \tilde{H}'_{\matA}(\inner{f_{\matA}(h \uwave{\tilde{u}})},
    v, \uwave{r}) &
    \text{where } w = u v \text{ for } u \in \bit^{b} \uwave{\text{ and }
    \tilde{u} = u \oplus r} \text{.}
  \end{cases}
\end{align}

The only differences between the salted-mode functions
$\tilde{H}_{\matA,e}, \tilde{H}'_{\matA}$ and their corresponding
unsalted-mode functions $H_{\matA,e}, H'_{\matA}$ are depicted above
with \uwave{wavy underlines}, and are as follows:
\begin{enumerate}
\item Each function $\tilde{H}_{\matA,e}, \tilde{H}'_{\matA}$ takes an
  additional salt input $r \in \bit^{b}$, whose length is the block
  length.

\item The function~$\tilde{H}_{\matA,e}(x,r)$ prepends an all-zeros
  block $0^{b} \in \bit^{b}$ to the input~$x$ (before padding it
  using~$\pad_{b,e}$ and inputting the result to the appropriate
  chaining function, as in the unsalted case). Note that this
  prepended zeros block counts toward the total input length in the
  padding function.

\item The chaining function~$\tilde{H}'_{\matA}$ XORs each block
  $u \in \bit^{b}$ of the padded input---including the initial,
  prepended all-zeros block---with the salt value $r \in \bit^{b}$
  (before inputting the result together with the previous chaining
  value~$h$ to the compression function, as in the unsalted case).
\end{enumerate}

% flatex input end: [arbitrary.tex]

% list of notes to address; appears only in draft mode

% flatex input: [instantiation.tex]

\section{Concrete Instantiations}
\label{sec:instantiations}

\newcommand{\xof}{\text{XOF}}
\newcommand{\pack}{\text{pack}}

The fully specified hash functions are merely instantiations of the
unsalted function~$H_{\matA,e}$ (\cref{eq:H}) and salted
function~$\tilde{H}_{\matA,e}$ (\cref{eq:H-tilde}) for specific
subset-sum parameters $q, n, m$, matrices
$\matA \in \Zq^{n \times m}$, and input-length representation bit
lengths~$e$.

\subsection{Deriving $\matA$}
\label{sec:deriving-A}

It is well known that for typical parameters, it is easy to generate a
matrix $\matA \in \Zq^{n \times m}$ that is indistinguishable from
uniformly random, together with some \emph{known collisions} in the
associated subset-sum compression function~$f_{\matA}$. This may allow
the party that generated~$\matA$ to violate the security of
constructions that use this function. Therefore, it is very important
to generate a random-looking~$\matA$ in a ``nothing up my sleeve''
manner that is highly unlikely to admit any such backdoor.

Let $q=2^{u}, n, m$ be subset-sum parameters where~$u$ is a positive
integer, and let $\xof \colon \bit^{*} \to \bit^{\infty}$ represent a
suitable cryptographic \emph{extendable-output function}, such as
SHAKE-256. Then for any identifier $id \in \bit^{*}$, define the
matrix $\matA_{\xof, id} \in \Zq^{n \times m}$ as:
\begin{equation}
  \label{eq:A_XOF}
  \matA_{\xof,id} := \pack_{u,n,m}\parens[\big]{ \xof
    \parens*{\inner{u}_{16}\, \inner{n}_{16}\,
      \inner{m}_{16}\, id}_{0, \ldots, unm-1}} \in \Zq^{n \times m} \text{,}
\end{equation}
where $\pack_{u,n,m} \colon \bit^{u n m} \to \Zq^{n \times m}$
constructs its output matrix from its input string in row-major order,
using~$u$ bits per entry. That is, for all $i = 0, \ldots, n-1$ and
$j = 0, \ldots, m-1$, the distinguished representative of the
$(i,j)$th entry~$a_{i,j} \in \Zq$ of $\matA = \pack_{u,n,m}(w)$ has
binary representation
\[ \inner{\bar{a}_{i,j}}_{u} = w_{u(im+j), \ldots, u(im+j) + (u-1)}
  \text{.} \]

\subsection{Concrete Parameters}
\label{sec:concrete-parameters}

The implemented functions are (unsalted) $H := H_{\matA,e}$ and
(salted) $\tilde{H} := \tilde{H}_{\matA,e}$, where:
\begin{itemize}
\item the modulus $q=2^{64}$, so $u = \log q = 64=2^{6}$;
\item the dimension $n = 8 = 2^{3}$, so the output length is
  $\ell = n \log q = 512 = 2^{9}$;
\item the input length $m = 1024 = 2^{10}$, so the block length is
  $b = m - \ell = 512 = 2^{9}$;
\item the representation length (of the hash input length) is $e = 128
  = 2^{7}$;
\item the extendable-output function is $\xof = \text{SHAKE-256}$;
\item the matrix is $\matA := \matA_{\xof,id}$ where $id$ is the ASCII
  representation of \verb=Algorand=.
\end{itemize}

% flatex input end: [instantiation.tex]

% list of notes to address; appears only in draft mode

%FLATEX-REM:\bibliography{common/lattices,common/crypto}
%*flatex input: [sumhash-spec.bbl]
\begin{thebibliography}{Dam89}
\expandafter\ifx\csname urlstyle\endcsname\relax
  \providecommand{\doi}[1]{doi:\discretionary{}{}{}#1}\else
  \providecommand{\doi}{doi:\discretionary{}{}{}\begingroup
  \urlstyle{rm}\Url}\fi

\bibitem[Ajt96]{ajtai04:_gener_hard_instan_lattic_probl}
M.~Ajtai.
\newblock Generating hard instances of lattice problems.
\newblock \emph{Quaderni di Matematica}, 13:1--32, 2004.
\newblock Preliminary version in STOC 1996.

\bibitem[Dam89]{DBLP:conf/crypto/Damgard89a}
I.~Damg{\aa}rd.
\newblock A design principle for hash functions.
\newblock In \emph{CRYPTO}, pages 416--427. 1989.

\bibitem[HK06]{DBLP:conf/crypto/HaleviK06}
S.~Halevi and H.~Krawczyk.
\newblock Strengthening digital signatures via randomized hashing.
\newblock In \emph{CRYPTO}, pages 41--59. 2006.

\bibitem[IN89]{DBLP:journals/joc/ImpagliazzoN96}
R.~Impagliazzo and M.~Naor.
\newblock Efficient cryptographic schemes provably as secure as subset sum.
\newblock \emph{J. Cryptology}, 9(4):199--216, 1996.
\newblock Preliminary version in FOCS 1989.

\bibitem[Mer89]{DBLP:conf/crypto/Merkle89}
R.~C. Merkle.
\newblock A certified digital signature.
\newblock In \emph{CRYPTO}, pages 218--238. 1989.

\end{thebibliography}

% flatex input end: [sumhash-spec.bbl]
%FLATEX-REM:\bibliographystyle{common/alphaabbrvprelim}

\end{document}


%%% Local Variables:
%%% mode: latex
%%% TeX-master: t
%%% End:

% flatex input end: [sumhash-spec.tex]
