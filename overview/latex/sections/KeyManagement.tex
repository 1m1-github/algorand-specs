\documentclass[../main.tex]{subfiles}

\begin{document}

\paragraph{Spending Keys.}

Algorand uses the Ed25519 signature scheme based on RFC 8032. 

\paragraph{Multi-signature Addresses.} 

Algorand supports threshold multi-signature addresses as follows. 
Suppose a user wants to create a threshold multi-signature address consisting of three keys, where at least two keys 
must co-sign any transaction. To accomplish this, the user must: 
\begin{itemize}
\item Sample 3 key pairs $(\PK1, \SK1), (\PK2, \SK2), (\PK3, \SK3)$. (These keys may be samples on independent devices).
\item Compute account public key: 
\[
\APK = Hash(\PK1, \PK2, \PK3, \textit{threshold}=2)
\]
\end{itemize}

Now, suppose the user has some tokens in the balance table associated with her unique APK. 
A valid transaction with respect to $\APK$ includes the hash pre-image of $\APK$ and signatures of the same transaction under at least 2 of the keys $(\PK1, \PK2, \PK3)$. 
The committee members in the consensus protocol check that the hash of the pre-image matches $\APK$ registered on chain, 
and that a sufficient number of signatures under the public keys are provided. 

\paragraph{Verifiable Random Function (VRF) Keys.} 

To participate in the consensus protocol, a user must register a VRF public key in her balance entry, with a corresponding public/secret key pair $(VRFPK,VRFSK)$. We use $VRF(x)$ to denote $VRFSK$'s signature of $x$.
Algorand follows the EC-VRF specifications outlined in draft-irtf-cfrg-vrf-04.\\

The VRF that Algorand uses is secure against malicious key registration. 
That is, even if a user maliciously samples and registers a $\VRFPK$ on chain, all outputs she obtains are still unique and cannot be biased.
This is accomplished partly because of the following: 
for a fixed $\VRFPK$, in round $r$ the VRF is applied on inputs $(Q^{r-2}, \APK)$, where $\VRFPK$ must have been in the blockchain by round $r-322$, and $\APK$ is the unique account public key of the user.



\paragraph{Participation Keys.} 

Algorand uses the Ed25519 digital signature scheme for participation keys. If a user $u$ wants to participate in the consensus protocol, 
she samples a key-pair $(\PartPK_u, \PartSK_u)$ and a collection of ephemeral keys for a range of rounds $[v,w]$
\[
(\EPK_u^v, \ESK_u^v, \ldots, \EPK_u^w, \ESK_u^w).
\]

Using $\PartSK_u$, she then signs each ephemeral key $\EPK_u^i$ and the corresponding round number $i$ to obtain a signature $\sigma_i$. 
After signing all ephemeral keys, the node deletes $\PartSK_u$ and stores the collection: 
\[
\big((\EPK_u^v, \ESK_u^v, \sigma_v), \ldots, (\EPK_u^w, \ESK_u^w, \sigma_w)\big).
\] 

We use $ESIG_u^r(x)$ to denote $ESK_u^r$'s signature of $x$.
To participate in the consensus protocol the user registers $\PartPK_u$ and the range $[v,w]$ for which it generated the ephemeral keys. 
Now, suppose a user is chosen by cryptographic self-selection to sign a message $M$ at round $r$. 
\begin{enumerate}
\item The user uses secret key $\ESK_u^r$ to sign $M$ to obtain signature $\sigma_M$. 
\item The user propagates $(\sigma_M, \EPK^r_u, \sigma_r)$ through the network.
\item The tuple ($\EPK^r_u, \sigma_r$) can be verified with respect to $\PartPK_u$ registered on chain to ensure that indeed $\ESK^r_u$ was authorized to sign messages for round $r$. 
\item After the user observes the confirmation of $B^r$, the user deletes $\ESK^r_u$. After deletion, the user can never sign any other message for round $r$. 
\end{enumerate} 


\end{document}